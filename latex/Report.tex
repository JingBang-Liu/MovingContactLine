\documentclass[
reprint,
%superscriptaddress,
%groupedaddress,
%unsortedaddress,
%runinaddress,
%frontmatterverbose, 
%preprint,
%preprintnumbers,
%nofootinbib,
%nobibnotes,
%bibnotes,
 amsmath,amssymb,
 aps,
%pra,
%prb,
%rmp,
%prstab,
%prstper,
%floatfix,
url
]{revtex4-1}

\usepackage{graphicx}% Include figure files
\usepackage{dcolumn}% Align table columns on decimal point
\usepackage{bm}% bold math
\usepackage{siunitx}
\usepackage{subcaption}
\usepackage{gensymb}
\usepackage[french,english]{babel}
\captionsetup{font={small},skip=0.25\baselineskip,justification=raggedright}
\captionsetup[subfigure]{font={small}, skip=1pt, singlelinecheck=false, justification=raggedright}
%\usepackage{hyperref}% add hypertext capabilities
%\usepackage[mathlines]{lineno}% Enable numbering of text and display math
%\linenumbers\relax % Commence numbering lines

%\usepackage[showframe,%Uncomment any one of the following lines to test 
%%scale=0.7, marginratio={1:1, 2:3}, ignoreall,% default settings
%%text={7in,10in},centering,
%%margin=1.5in,
%%total={6.5in,8.75in}, top=1.2in, left=0.9in, includefoot,
%%height=10in,a5paper,hmargin={3cm,0.8in},
%]{geometry}

\begin{document}

\preprint{APS/123-QED}

\title{Thermal Fluctuations of Free Surface and Contact Angle: \\a Molecular Dynamics Approach}% Force line breaks with \\


\author{Jingbang Liu}
 \email{jingbang.liu@warwick.ac.uk}
\affiliation{%
 Mathematics Institute, University of Warwick\\
 Jingbang.Liu@warwick.ac.uk
}%


\author{James E. Sprittles}
\email{j.e.sprittles@warwick.ac.uk}
\affiliation{
 Mathematics Institute, University of Warwick
}%

\author{Duncan A. Lockerby}
\email{duncan.lockerby@warwick.ac.uk}
\affiliation{%
 School of Engineering, University of Warwick
}%


\date{\today}% It is always \today, today,
             %  but any date may be explicitly specified

\begin{abstract}
In this report large-scale molecular dynamics simulations have been carried out to study the fluctuations in the free surface position and contact angle. We confirm that these fluctuations can be modelled as a Langevin process and their statistics obey Gaussian distribution as reported in earlier papers. We further show that after applying the Bayesian linear regression to determine the trust region of exponential decay, the coefficient of friction can be extracted by fitting the autocorrelations of the fluctuations to exponential functions. Moreover, we find that the free surface fluctuates more near the solid plate and a theory based on lubrication approximation of 2D Navier-Stokes equations to explain this phenomenon is under development. 

\begin{description}
\item[keywords]
Contact line, Contact angle, Thermal fluctuation, Langevin process, Molecular Dynamics
\end{description}
\end{abstract}

%\keywords{Suggested keywords}%Use showkeys class option if keyword
                              %display desired
\maketitle

%\tableofcontents

\section{Introduction}

Wetting is a common phenomenon that describes the spreading of liquids on solid surfaces. The study of it can be dated back to the 19th century when the famous Young-Laplace equation \cite{young_iii_1805} was proposed. By equating the pressure difference to the surface tension originated from curvature, Young and Laplace established a continuum model to predict equilibrium contact angle, which is the angle between the solid liquid interface and liquid fluid interface, for liquid and homogeneous solid. Since then the macroscopic properties of wetting have drawn much attention and have been well investigated.  Recently due to the development of applications related to microfluidics and nanotechnology \cite{li_wetting_1994}, dynamic wetting with moving contact line (i.e. the line where liquid-fluid-solid meet) in the microscopic scale has become a subject of much interest. However when adapting the conventional approaches to model the flow near the moving contact line a shear-stress singularity emerges \cite{shikhmurzaev_moving_1997}. This is the famous ``moving contact line problem" and to address it, additional physics such as a Navier slip condition between the liquid and solid must be introduced \cite{td_blake_ydshikhmurzaev_dynamic_2002}.

Experiments have been conducted in order to gather information about what additional physics to include and validate assumptions \cite{chen_convex_2014}. But experiments at such a small scale are often hard to perform and observe, and varying parameters independently to establish their influence is usually impossible. In this case computer simulations provides an alternative solution. Thanks to the fast development in computing power we are now able to run large scale molecular dynamics (MD) simulations in an affordable way. MD simulation has the advantage of allowing us to look at dynamic wetting problems from a molecular perspective and naturally incorporate molecular phenomena such as thermal fluctuations \cite{bertrand_dynamics_2007,bertrand_influence_2009,shang_fluctuating_2011,smith_langevin_2016,fernandez-toledano_molecular-dynamics_2019,fernandez-toledano_contact-line_2019}. It is also capable of reproducing liquid properties such as shear viscosity \cite{hess_determining_2001} and self diffusion coefficient \cite{easteal_self-diffusion_1983} which adds credibility to its results. Despite successful applications above, two drawbacks of MD are obvious: (i) the potentials that can be applied to the molecules are still limited, making it hard to compare with real-world experiments. (ii)  although a few tens of thousands of molecules can be simulated, it is still in the scale of nanometres and nanoseconds \cite{terence_d_blake_physics_2006}. Since this work only focus on the theoretical aspect of moving contact line and contact angle, current MD simulations at the nanoscale is enough.  

Molecules throughout the liquid domain are in constant motion and agitation due to thermal effects. The statistical properties of these fluctuations often contain important information that could be linked to the macroscopic behaviour of the free surface. Recently some researchers successfully used the Langevin model, which was originally used to model the behaviour of Brownian particles, to capture the thermal fluctuations in the contact angle \cite{smith_langevin_2016} and the contact line \cite{fernandez-toledano_contact-line_2019}. Moreover, they showed that the statistical properties of the fluctuation can be used to calculate the friction parameter in a long existing model \cite{cherry_kinetics_1969, blake_kinetics_1969}, namely the molecular kinetic theory. In this report we will use large scale MD simulations to model a tank of liquid with a free surface. We then extract the position of free surface  and calculate the contact angle to show that both time series fit well into the Gaussian distribution. We show that the amplitude of fluctuations of free surface is related to its distance to the solid plates. In the end we show that the free surface and the contact angle can be modelled as Langevin processes to extract the coefficient of friction.



\section{Methods}

In this section we will layout the settings for MD simulation and explain the methods used to calculate the free surface position, the contact line position and the contact angle. We will also explain how to use the Langevin equation to model the fluctuations.

\subsection{Molecular Dynamics Simulation}

Molecular dynamics simulations are used here as an tool of experiment to measure the fluctuation of the three-phase contact line. As shown in Fig.~\ref{fig:simulation2}, the set up of the simulation is similar to a tank of liquid. The bulk of liquid is placed between two parallel solid plates which forms two contact lines on the left and the right. Different from the configuration used in Blake's paper \cite{fernandez-toledano_contact-line_2019}, we add a third solid plate to prevent the liquid's centre of mass from moving. Periodic boundary conditions are applied in the $y$ direction to allow liquid particles to pass.  Simulations are performed in LAMMPS package \cite{plimpton_fast_1995}, which has been widely used to study fluid phenomena in nanoscale \cite{blake_forced_2015,zhang_molecular_2019,zhao_revisiting_2019}.

The interaction of particles are modelled using the Lennard-Jones $12-6$ potential:
\begin{equation}
	V(r_{ij}) = 4\epsilon C_{AB}\bigg[\big(\frac{\sigma}{r_{ij}}\big)^{12} - \big(\frac{\sigma}{r_{ij}}\big)^6\bigg],
\end{equation}
where $r_{ij}$ is the distance between atoms $i$ and $j$, $\epsilon$ is the energy parameter representing the depth of potential wells, $\sigma$ is the length parameter representing the effective atomic diameter and $C_{AB}$ is a non-dimensional coupling parameter that allows us to adjust the strength of coupling between different types of particles $A$ and $B$, namely solid-solid (S-S), liquid-liquid (L-L) and solid-liquid (S-L). In our simulation for both liquid and solid particles $\sigma = 3.5$ \si{\angstrom} and $\epsilon = k_B T$, where $k_B$ is the Boltzmann constant and $T=33$ K is the temperature. This choice of temperature ensures the fluid remains liquid. $C_{SS}$ and $C_{LL}$ are both given value of $1.0$, while $C_{SL}$ is varied from $0.6$ to $0.9$ to study its effect on the results. A cutoff radius $\sigma_c = 2.5\sigma$ is applied so interactions will only be considered if the $r_{ij}\leq \sigma_c$. Both mass of liquid and solid particles are set to be $12$ g/mol to control variates.

\begin{figure}
	\includegraphics[scale=0.6]{simulation2.png}
	\caption{\label{fig:simulation2}Instantaneous snapshot of the  molecular dynamics simulation with solid and liquid particles painted in blue and red.}
\end{figure}

The simulation box has dimensions $(L_x,L_y,L_z) = (137.2,121.45,168.7)$ \si{\angstrom}. Each solid plate is smooth on the surface and has three layer of solid particles sitting in cubic lattices. The bottom plate has $35\times 31\times 3=3255$ atoms and the side plates have $3\times 31 \times 40=3720$ atoms each. The lattice constant is chosen to be $2^{1/6} \sigma\approx 3.93$ \si{\angstrom} which is the distance of equilibrium for two solid particles under Lennard-Jones $12-6$ potential. An additional harmonic potential:
\begin{equation}
	V_h(r) = \frac{1000\epsilon}{\sigma^2}|\mathbf{r}-\mathbf{r}_0|^2,
\end{equation} 
is applied to the solid particles allowing them to oscillate around their initial positions to exchange momentum with the liquid while maintaining the shape of plate. Here $\mathbf{r}$ is the instantaneous position of the particle and $\mathbf{r}_0$ is the initial position of the particle. The liquid is comprised of $29760$ particles. Every $8$ liquid particles are chained by a finitely extensible nonlinear elastic (FENE) bond with potential
\begin{equation}
	V_{\textit{FENE}}(r_{ij}) = -0.5\alpha R_0^2 \ln(1-\frac{r_{ij}^2}{R_0^2}),
\end{equation}
where the strength $\alpha = 0.04556$ J/m$^2$ and maximum bond length $R_0 = 1.4\sigma$ \cite{fernandez-toledano_contact-line_2019}. The FENE bonds help to avoid undesirable breakups and create vacuum conditions in the gas phase, so that we can ignore its influence on the dynamics.

Some liquid properties are calculated to compare with other literatures. Shear viscosity $\nu=0.1362$ mPa$\cdot$s is calculated using the Green-Kubo method \cite{haile_molecular_1993}:
\begin{equation}
	\nu = \frac{V}{3k_B T}\sum\int_0^{\infty}\langle \mathbf{J}_{\alpha\beta}(t)\mathbf{J}_{\alpha\beta}(0)\rangle dt,
\end{equation}
where $V$ is the volume, $\mathbf{J}$ is the stress tensor, $\langle$ $\rangle$ calculates the autocorrelation and the sum accumulates three terms given by $\alpha\beta(=xy,yz,zx)$. Surface tension $\gamma = 2.83$ mN/m is calculated from the difference between the normal and tangential component of pressure tensor in a simple vapour-liquid-vapour system \cite{trokhymchuk_computer_1999}:
\begin{equation}
	\gamma = \frac{1}{2}\int_0^{L_z}[\mathbf{P}_n(z)-\mathbf{P}_t(z)]dz,
\end{equation} 
where $L_z$ is the length of the simulation box, $\mathbf{P}_n$ and $\mathbf{P}_t$ are the normal and tangential component of the pressure tensor respectively. 

The time step used is $0.005$ ps. At the beginning of each run the system is equilibrated for $10$ ns ($2\times10^6$ time steps) using NVE (a micro-canonical ensemble where the number of particles, the system's volume and the total energy are kept constant) to achieve a stable configuration. The simulation then run for another $100$ ns ($2\times 10^7$ time steps) using NVT (a micro-canonical ensemble where the number of particles, the system's volume and the temperature are kept constant).

\subsection{Contact Line Positions and Contact Angles}

The positions of contact lines can be considered as the positions where the free surface touches the solid plates. To simplify the problem, instead of treating the free surface as a 2D object, we study its averaged projection onto the x-z plane $z_f(x,t)$. Therefore the left and right contact line positions $z_{cl}^l(t)$ and $z_{cl}^r(t)$ are no longer 'lines' but 'points', and are obtained as follows. The simulation results shows that for $C_{SL}=0.6,0.7,0.8,0.9$ the free surface always stay in the cuboid region $\Omega_f=\{(x,y,z)|x\in[10.15,124.6],y\in[0.0,121.45],z\in[105.0,157.5]\}$. $\Omega_f$ is then divided into $32\times30\times15=14400$ bins to calculate the number density of liquid particles. Each bin has the dimension of $(1.0219\sigma,1.147\sigma,\sigma)$, which is small enough to capture fluctuations yet to yield stable results. The number density is then averaged in $y$ direction to give a density profile projected onto the $x-z$ plane. As shown in Fig.~\ref{fig:fittings} (a) the density in bulk $D_b$ is almost constant and decreases to zero in vacuum. The position of the free surface is then said to be where the density decays to $0.5 D_b$, and can be represented as a discrete function $z_f(x)$ of $x\in X$ where $X$ is a set of all $x$ coordinate of the centre of the bins. We then fit the density profiles into a modified hyperbolic tangent function $T(z)$ as shown below.
\begin{equation}
	\label{eq:tanh}
	T(z) = \frac{D_b}{2}\big[\frac{\exp(a(z-b))-1}{\exp(a(z-b))+1}+1\big]
\end{equation}
\begin{figure*}
	\centering
	\begin{subfigure}[b]{0.49\linewidth}
		\caption{}
		\includegraphics[width=\linewidth, keepaspectratio]{densityProfile.png}
	\end{subfigure}	
	\begin{subfigure}[b]{0.49\linewidth}
		\caption{}
		\includegraphics[width=\linewidth]{CLfit.png}
	\end{subfigure}
	\begin{subfigure}[b]{0.49\linewidth}
		\caption{}
		\includegraphics[width=\linewidth]{CAfit1.png}
	\end{subfigure}
	\begin{subfigure}[b]{0.49\linewidth}
		\caption{}
		\includegraphics[width=\linewidth]{CAfit2.png}
	\end{subfigure}
	\caption{\label{fig:fittings} Fitting to instantaneous contact line position and contact angle. The coupling coefficient is $C_{sl}=0.9$. (a) shows the density profile of the liquid after being averaged in the $y$ direction. In (b) we fit the density profile of the liquid layer nearest the left (west) wall using a hyperbolic tangent function as shown in Eq.~\ref{eq:tanh} to calculate the contact line position $z_{cl}$. In (c) and (d) we fit the free surface position to parabolic functions as shown in Eq.~\ref{eq:parabolic} to calculate the left and right contact angles, $\theta_l$ and $\theta_r$. Each fitting uses $15$ points. Results are $\theta_l=67.19\degree$ and $\theta_r=66.96\degree$. Note in the figure the scale of $x$ is larger than the scale of $z$. }
\end{figure*}
Here $D_b$ is the bulk density, $a$ and $b$ are fitting parameters. $T(z)$ is a monotonic function with maximum $D_b$ and minimum $0$ which is a property of the density profile. Fig.~\ref{fig:fittings}(b) gives an example of the fitting and shows that $T(z)$ describes the density profile well. One can easily see that for $x'\in X$, $z_f(x') = b$ since $T(b)=D_b/2$. We repeat this process for all $x\in X$ and time steps to get the time series of free surface $z_f(x,t)$, $z_{cl}^l(t)$ and $z_{cl}^r(t)$.   


The contact angle is defined as the angle between the liquid-gas interface and the liquid-solid interface at a common point, which is equivalent to the angle between the free surface and the wall. Different methods have been use to measure contact angle from observation. Bo Shi and Vijay K. Dhir in their paper \cite{shi_molecular_2009} captured the free surface by analysing the density profile and calculated the contact angle by drawing a straight line that overlaps the curved interface near the contact line (within 3$\sigma$). Instead of only taking a small part of the free surface near the contact line, Blake and others \cite{blake_contact_1997,blake_forced_2015,fernandez-toledano_molecular-dynamics_2019} ignored the first three layers of liquid particles near the solid plate and fitted the whole interface with a circle to calculate the contact angle as this is an equilibrium profile in the absence of fluctuations. Smith et al. \cite{smith_langevin_2016} took an intermediate approach by fitting half of the free surface using polynomial functions. Since Smith et al.'s simulation set up is similar to ours, we will adapt their approach. The left contact angle $\theta^l$ and the right contact angle $\theta^r$ are treated separately. For each contact angle we fit $z_f(x)$ with the parabolic functions $P(x)$,
\begin{equation}
	\label{eq:parabolic}
	P(x) = cx^2+dx+e,
\end{equation}
where $c$, $d$ and $d$ are fitting parameters. Let $x_l$ and $x_r$ be the left point and the right point of the free surface domain, the two contact angles are then calculated by
\begin{align}
	\theta_l &= \frac{\pi}{2} + \arctan(2c_lx_l+d_l), \\
	\theta_r &= \frac{\pi}{2} - \arctan(2c_rx_r+d_r),
\end{align} 
where the codomain of $\arctan$ is given as $[-\pi/2,\pi/2]$. Note that one key difference between varies contact angle calculation methods is the number of points $N_f$ on the free surface (i.e. the proportion of free surface) used in fitting. Its affect is also examined and the results are shown in the next section. Fig.~\ref{fig:fittings}(c)(d) shows that, although simple, the parabolic function is able to sufficiently describe the shape of the free surface.

\begin{figure*}[t]
	\centering
	\begin{subfigure}[b]{0.45\linewidth}
		\caption{}
		\includegraphics[width=\linewidth]{CL_distribution1.png}
	\end{subfigure}
	\begin{subfigure}[b]{0.45\linewidth}
		\caption{}
		\includegraphics[width=\linewidth]{CL_distribution2.png}
	\end{subfigure}
	\begin{subfigure}[b]{0.45\linewidth}
		\caption{}
		\includegraphics[width=\linewidth]{CL_mean.png}
	\end{subfigure}	
	\begin{subfigure}[b]{0.45\linewidth}
		\caption{}
		\includegraphics[width=\linewidth]{CL_std.png}
	\end{subfigure}
	\caption{\label{fig:CL} Results Gaussian distribution fitting of the time series of free surface position $z_f$ with different coupling parameter $C_{SL}=0.5$-$0.9$. The method of fitting is trust region reflective algorithm and error bars are obtained accordingly. (a) shows the time series of $z_f(x)$ at $x=10.15,124.6,65.58$ \si{\angstrom} and (b) shows the probability density functions after fittings them to the Gaussian distribution with $C_{SL}=0.9$. (c) and (d) show the change of the mean free surface position $\mu_f$ and the standard deviation of free surface position $\sigma_f$ with $x$ for different $C_{SL}$.}
\end{figure*}

\subsection{The Langevin Model}

The Langevin model is a phenomenological equation which was first constructed on known macroscopic laws to describe the motion of Brownian particles as a stochastic process \cite{kampen_stochastic_2007}. Recently it has been successfully used to capture thermal fluctuations in  contact line position \cite{fernandez-toledano_contact-line_2019} and contact angle \cite{smith_langevin_2016}. The Langevin equation is usually displayed in the following form \cite{feldman_theory_1989}. Suppose $x$ is a quantity of interest that varies with time, then
 
\begin{figure*}[t]
	\centering
	\begin{subfigure}[b]{0.45\linewidth}
		\caption{}
		\includegraphics[width=\linewidth]{CA_distribution1.png}
	\end{subfigure}
	\begin{subfigure}[b]{0.45\linewidth}
		\caption{}
		\includegraphics[width=\linewidth]{CA_distribution2.png}
	\end{subfigure}
	\begin{subfigure}[b]{0.45\linewidth}
		\caption{}
		\includegraphics[width=\linewidth]{CA_mean.png}
	\end{subfigure}	
	\begin{subfigure}[b]{0.45\linewidth}
		\caption{}
		\includegraphics[width=\linewidth]{CA_std.png}
	\end{subfigure}
	\caption{\label{fig:CA} Results of Gaussian distribution fitting of the time series of contact angles with different $N_f$ on the free surface being considered. Results of left contact angle $\theta_l$ and right contact angle $\theta_r$ are shown separately. The method of fitting is nonlinear least square and error bars are obtained accordingly. (a) shows the time series of $\theta_l$ and $\theta_r$ and (b) shows the probability density functions after fitting them to the Gaussian distribution with $C_{SL}=0.9$. (c) and (d) show the change of the mean contact angle $\mu_{\theta}$ and the standard deviation of contact angle $\sigma_{\theta}$ with $N_f$ accordingly.}
\end{figure*}

\begin{equation}
	\label{eq:langevine}
	\xi\frac{d x}{dt} = -kx + \eta(t),
\end{equation}
where $\xi$ is the coefficient of friction, $k$ is the damping factor and $\eta(t)$ is the noise term representing collective effect of random forces that satisfies
\begin{equation}
	\label{eq:delta}
	\langle \eta(t) \rangle = 0,
\end{equation}
meaning that their effect on $\langle x\rangle$ should cancel out given sufficiently long time and,
\begin{equation}
	\label{fluc}
	\langle \eta(t)\eta(t') \rangle = 2\xi k_B T\delta(t-t'),
\end{equation}
meaning  that the forces are uncorrelated at different time. The factor of $2\xi k_B T$ is the result of the fluctuation-dissipation theorem \cite{feldman_theory_1989}. $\langle$ $\rangle$ is the ensemble average and $\delta(t)$ is the Dirac delta. By Eq.~\ref{eq:langevine} one can solve for $x(t)$ in terms of $\eta(t)$
\begin{equation}
	\label{eq:sol_lang}
	x(t) = \frac{1}{\xi}\int^t_{-\infty} e^{-(t-t')k/\xi}\eta(t')dt'.
\end{equation}
Combining Eq.~\ref{eq:delta} and Eq.~\ref{eq:sol_lang} one can derive that the autocorrelation function of $x$ should decay with $t$ exponentially \cite{feldman_theory_1989}.
\begin{equation}
	\label{eq:decay}
	\langle x(t+t_0)x(t_0) \rangle = \frac{k_B T}{k}e^{-kt/\xi}
\end{equation}
Let $x = z_{cl}$, we can then calculate $\langle z_{cl}(t+t_0)z_{cl}(t_0) \rangle$ from MD simulation and fit it to an exponential function to obtain the exponent $\alpha = k_{cl}/\xi_{cl}$. Note that for $t=0$,
\begin{equation}
	\langle z_{cl}(t_0)z_{cl}(t_0) \rangle = k_BT/k_{cl} = \sigma_{cl}^2.
\end{equation}
So
\begin{equation}
	\label{eq:xiTl}
	\xi_{cl} = \frac{k_BT}{\alpha\sigma_{cl}^2},
\end{equation}
can be calculated from MD simulation.
Similarly for the contact angle $\theta$ we should have
\begin{equation}
	\label{eq:xiTheta}
	\xi_{\theta} = \frac{k_BT}{\beta\sigma_{\theta}^2},
\end{equation}
where $\beta=k_\theta/\xi_\theta$ is obtained from MD simulation.




\section{Results and Discussion}

In this section we show that the fluctuations of the free surface and contact angle follow the Gaussian distribution. The amplitude of fluctuation of the free surface is greater near the solid plate. These fluctuations can be modelled by the Langevin equation and the friction parameters can be calculated accordingly.

\begin{figure*}[t]
	\centering
	\begin{subfigure}[b]{0.45\linewidth}
		\caption{}
		\includegraphics[width=\linewidth]{CL_auto.png}
	\end{subfigure}	
	\begin{subfigure}[b]{0.45\linewidth}
		\caption{}
		\includegraphics[width=\linewidth]{CA_auto.png}
	\end{subfigure}
	\begin{subfigure}[b]{0.45\linewidth}
		\caption{}
		\includegraphics[width=\linewidth]{CL_xi.png}
	\end{subfigure}
	\begin{subfigure}[b]{0.45\linewidth}
		\caption{}
		\includegraphics[width=\linewidth]{CA_xi.png}
	\end{subfigure}
	\caption{\label{fig:auto}(a) and (b) show the time evolution of $\langle z_{cl}(t+t_0)z_{cl}(t_0)\rangle/\langle z_{cl}^2(t_0)\rangle$ and $\langle \theta(t+t_0)\theta(t_0)\rangle/\langle \theta^2(t_0)\rangle$ for $C_{SL}=0.5$-$0.9$. (c) and (d) show $\xi_{cl}$ and $\xi_{\theta}$ for different $C_{SL}$.}
\end{figure*}

\subsection{Fluctuation of Free Surface and Contact Angle}

We run MD simulations with $C_{SL}=0.5$-$0.9$. The number density profile of liquid particles in the free surface region $\Omega_f$ is collected from the MD simulation every $1000$ time steps so the size of the data files are manageable and information in fluctuation is not lost. This leads to $20000$ individual snapshots of the number density profile of liquid being recorded for each run. The time step between each snapshot is $0.005$ ns and the mean displacement of the liquid particles between each time step is around $0.021$ \si{\angstrom} which is much smaller than the length parameter $\sigma=3.5$ \si{\angstrom}. The free surface position $z_f$ and contact angles $\theta_l$ $\theta_r$ are calculated according to methods described in the last section.

We find that the time series of $z_f$ can be fitted with the Gaussian distribution,
\begin{equation}
	f(z_f) = \frac{1}{\sqrt{2\pi \sigma^2_f}}\exp\big(-\frac{(z_f-\mu_f)^2}{2\sigma_f^2}\big).
\end{equation}
The mean $\mu_f$ and standard deviation $\sigma_f$ are obtained by fitting the probability density function to $f(z_f)$ using the trust region reflective algorithm from the SciPy package \cite{virtanen_scipy_2020}. Errors in the estimation of $\mu_f$ and $\sigma_f$ are generated automatically. From Fig.~\ref{fig:CL}(c) one can see shape of the free surface at equilibrium for different $C_{SL}$ and the trend that $\theta$ decreases with $C_{SL}$. Fig.~\ref{fig:CL}
(d) demonstrates that $\sigma_f$ is much smaller in the middle than on the side. This means that the free surface fluctuates more near the solid plates than the centre. Since $C_{LL}>C_{SL}$ one could argue that liquid particles are less constrained near the solid plates leading to $z_f$ fluctuates more. However this means that $z_f$ should fluctuates more for small $C_{SL}$ which is missing from Fig.~\ref{fig:CL}(d). We can also take a continuum approach to explain this phenomenon. By studying linear stability of the lubrication approximation of 2D Navier-Stokes equations \cite{acheson_elementary_1990} we found that the variance of the perturbation in height is maximised at the boundary and minimized at the centre \footnote{This is preliminary work still under development.}, which agrees with our observation in the MD simulation. To formalise this theory further investigation is required.


We find that the time series of $\theta_l$ and $\theta_r$ can also be well fitted into the Gaussian distribution as shown in Fig.~\ref{fig:CA}. The observed amplitude of fluctuation in $\theta_l$ and $\theta_r$ is greater than what Fern\'andez et al. reported in their paper \cite{fernandez-toledano_contact-line_2019}. This is likely to be caused by the different measuring methods. They simulated a large liquid drop on a solid surface, ignored the first $5$ layers next to the solid surface where the density profile is perturbed and accepted the best circular fit within the region of density dropped to $0.75-0.25$. A similar amplitude of fluctuation in contact angle was found by Smith et al. \cite{smith_langevin_2016} where they applied a polynomial fit and considered the layers next to the solid surface. Fig.~\ref{fig:CA}(c) shows that $N_f$ has little affect on the mean contact angles $\mu_{\theta}$ and the numbers agree well with Fern\'andez et al.'s \cite{fernandez-toledano_contact-line_2019} paper for $C_{SL}=0.7,0.8$. Fig.~\ref{fig:CA}(d) shows that the standard deviation of contact angles $\sigma_{\theta}$ decreases with $N_f$, meaning the more points we include in the calculation of $\theta_l$ and $\theta_r$ the less fluctuation we see in the results. This agrees with our previous observation in $\sigma_f$: $z_f$ is more stable in the middle, so the more points in the middle region are considered the less fluctuation in the shape of free surface to fit. Another possible explanation is that larger $N_f$ is more capable of capturing the quadratic term in $P(x)$, thus applying more constrain. 

\subsection{Fitting to the Langevin Model}


Time series of $z_{cl}$ and $\theta$ obtained from MD simulations are vectors of size $N=20000$. For $\theta$ we choose $N_f=18$ for a smoother autocorrelation function. Suppose $0\leq T\leq N-1$ is an integer index represents timestep, then
\begin{align}
	\langle z_{cl}(T+t_0)z_{cl}(t_0)\rangle &= \frac{1}{N-T}\sum_{i=1}^{N-T}z_{cl}(i)z_{cl}(i+T),\\
	\langle \theta(T+t_0)\theta(t_0)\rangle &= \frac{1}{N-T}\sum_{i=1}^{N-T}\theta(i)\theta(i+T)
\end{align}  
are also vectors of size $N=20000$. The two autocorrelation vectors for $z_{cl}$ are then normalised and averaged to smooth the curves. The same treatment is also applied on the two autocorrelation vectors for $\theta$. To perform fitting, we perform a Bayesian linear regression to determine the region in which the autocorrelation vectors exhibits exponential decay. Details of the scheme can be found in the Appendix. We then fit the normalised and averaged autocorrelation to an exponential function with the trust region reflective algorithm.

One can see from Fig.~\ref{fig:auto}(a) that MD simulation results support our assumption of $\langle z_{cl}(t+t_0)z_{cl}(t_0)\rangle$ decays exponentially with time. Also the decay rate $\alpha$ decreases monotonically as $C_{SL}$ increases. According to Eq.~\ref{eq:xiTl} this is saying that $\xi_{cl}$ increases with $C_{SL}$. This agrees with Fern\'andez et al.'s finding \cite{fernandez-toledano_contact-line_2019}. One possible explanation is that higher $C_{SL}$ applies higher constrain on liquid particles, causing the friction coefficient $\xi_{cl}$ to increase. Note that a similar argument has been used to explain the change of $\sigma_f$ with $x$. Fig.~\ref{fig:auto}(b) shows that $\langle \theta(t+t_0)\theta(t_0)\rangle$, although not perfectly smooth, also exhibits exponential decay with time. The decay rate $\beta$ also decreases as $C_{SL}$ increases. The roughness is probably caused by insufficient simulation runtime. Fig.~\ref{fig:auto}(c) shows the trend of $\xi_{cl}$. It agrees well with Fern\'andez al.'s finding \cite{fernandez-toledano_contact-line_2019}: the friction coefficient increases as the coupling between solid and liquid gets stronger. 

\subsection{Conclusion and Future Work}

In this work we used molecular dynamics simulation to study the fluctuations of the free surface position $z_f$ and the contact angle $\theta$. Lennard-Jones $12$-$6$ potential was applied and liquid particles are arranged in 8-particle chains by FENE bonds to avoid a vapour phase. Shear viscosity and surface tension of the liquid are calculated from MD simulation and compared with other literature. Fitting tools have been developed to locate free surface position and measure contact angle. We found that the less number of points on the free surface $N_f$ we use to calculate $\theta$, the noisier the measurements are. We confirmed that the fluctuations of $z_f$ and $\theta$ follows Gaussian distribution for all solid-liquid coupling parameters $C_{SL}=0.5$-$0.9$. The results show that $z_f$ fluctuates more near the solid plates. This is probably because the additional harmonic potential applied on the solid particles is passed onto near by liquid particles. The Langevin process was used to model the fluctuations in $z_{cl}$ and $\theta$. Autocorrelations $\langle z_{cl}(t+t_0)z_{cl}(t_0)\rangle$ and $\langle \theta(t+t_0)\theta(t_0)\rangle$ were obtained from MD simulation. After applying the Bayesian linear regression we found that the autocorrelations decay exponentially with time, which agrees with what the Langevin model predicted. 

In the future we would like to further investigate the cause for greater standard deviation of free surface position $\sigma_f$ near the solid plate. From a molecular point of view we can change the mass of particles to tweak the effect of forces passed on to liquid from solid. This way we hope to amplify the effect of $C_{SL}$ and examine its possible influence. From a continuum point of view we wish to adapt the Navier-Stokes equations into desirable forms. So far the lubrication approximation confirms that the fluctuation is maximised at the boundary and its amplitude is derived in terms of the surface tension and the dimensions of the thin film. So the next step is to modify the MD simulation to simulate a thin film scenario and compute $\sigma_f$ to find agreements quantitatively. Further more, we can use the stochastic lubrication equation \cite{landau_fluid_1995} to make the theory more compact. 





\begin{acknowledgments}
The authors acknowledge the use of the Scientific Computing Research Technology Platform, and associated support services at the University of Warwick, in the completion of this work.
\end{acknowledgments}

\appendix

\section{Supplementary Material}

Supplementary materials for details to examine the exponential decay of normalised autocorrelations can be found online at \url{https://github.com/JingBang-Liu/MiniProject.git}.

\section{Bayesian Linear Regression}

Suppose we are given a noisy time series $x$ of length $N$ and we are told that in theory its normalised autocorrelation function should decay exponentially with time for some unknown decay rate $\alpha$,
\begin{equation}
	A(t) = \frac{\langle x(t+t_0)x(t_0)\rangle}{\langle x(t_0)^2\rangle} \approx e^{-\alpha t}.
\end{equation}
Since $x$ has finite size, $A(t)$ will only be smooth and fit the exponential decay at its beginning. The rest of $A(t)$ will have a negative effect on the accuracy of $\alpha$ if we are to predict it by fitting the whole $A(t)$ to an exponential function. So we apply Bayesian linear regression to $A(t)$ to calculate a trust region in which we believe $A(t)$ is decaying exponentially with $t$. Detailed procedure is listed below \cite{kermode_james_px914_2020}.

First we take the logarithm of $A(t)$
\begin{equation}
	\log(A(t))\approx -\alpha t.
\end{equation}
The idea is to fit $\log(A(t))$ with a linear basis that starts from $0$
\begin{equation}
	\log(A(t)) \approx \mathbf{w}^T\mathbf{\phi}(t),
\end{equation}
where $\mathbf{\phi}=(0,t)^T$ is the basis and $\mathbf{w}=(w_1,w_2)^T$ is the weight. Let $\mathbf{t}$ be the time vector of size $N$. We then make a further assumption that $\log(A(\mathbf{t}))$ can be describe as $\mathbf{w}^T\mathbf{\phi}(\mathbf{t})$ with a multi-variant Gaussian noise of variance $\sigma_A$. Then we can write the probability of observing $\log(A(\mathbf{t}))$ at $\mathbf{t}$ with parameter $\mathbf{w}$ as
\begin{align}
	\notag
	\mathbb{P}(\log(A(\mathbf{t}))&|\mathbf{t},\mathbf{w},\sigma_A) = \\
	&(2\pi)^{-\frac{N}{2}}\sigma_A^{-N}e^{-\frac{1}{2\sigma_A^2}||\mathbf{w}^T\mathbf{\phi}(\mathbf{t})-\log(A(\mathbf{t}))||^2}.
\end{align}
Consider a prior on $\mathbf{w}$ which is also Gaussian
\begin{equation}
	\mathbb{P}(\mathbf{w}|\sigma_\mathbf{w}) = \mathcal{N}(\mathbf{w}|0,\sigma_\mathbf{w}^2\mathbf{I})
\end{equation}
where $\mathbf{I}$ is identity matrix. We can then apply Bayes rule to get the posterior of the $\mathbf{w}$
\begin{align}
	\notag
	\mathbb{P}(\mathbf{w}|\mathbf{t}&,\log(A(\mathbf{t})),\sigma_A,\sigma_\mathbf{w}) = \\
	&\frac{\mathbb{P}(\log(A(\mathbf{t}))|\mathbf{t},\mathbf{w},\sigma_A)\mathbb{P}(\mathbf{w}|\sigma_\mathbf{w})}{\int\mathbb{P}(\log(A(\mathbf{t}))|\mathbf{t},\mathbf{w}',\sigma_A)\mathbb{P}(\mathbf{w}'|\sigma_\mathbf{w})d\mathbf{w}'}
\end{align}
We then optimize $\sigma_A$ and $\sigma_\mathbf{w}$ with \textit{BayesianRidge.fit} from \textit{sklearn} package. In the end we get\cite{kermode_james_px914_2020}
\begin{align}
	\mathbb{P}(\log(A)'&|t',\mathbf{t},\log(A(\mathbf{t})),\sigma_A,\sigma_\mathbf{w})=\\
	&\mathcal{N}(\log(A)'|m(t'),s^2(t')),
\end{align}
where $t'$ is some time, $\log(A)'$ our new prediction about $\log(A)$ at $t'$,
\begin{equation}
	m(t') = \big[\sigma^2_A(\sigma^2_A\phi^T(\mathbf{t})\phi(\mathbf{t})+\sigma^2_\mathbf{w}\mathbf{I})^{-1}\phi^T(\mathbf{t})\log(A(\mathbf{t}))\big]^T\phi(t'),
\end{equation}
and 
\begin{equation}
	s(t')=\phi^T(t')(\sigma^2_A\phi^T(\mathbf{t})\phi(\mathbf{t})+\sigma^2_\mathbf{w}\mathbf{I})^{-1}\phi(t')+\frac{1}{\sigma^2_A}.
\end{equation}
This tells us that our new prediction is also a Gaussian distribution and at $t'$ the $\log(A)'$ can be modelled by the linear basis with confidence region $\pm 2s(t')$. Note that since $\log(A(0))$ is always $0$, $s(t')$ is monotonic. So $s(t')$ reflects how confident we are that up to $t'$ $A(\mathbf{t})$ decays with time exponentially. Also note that the number of samples (i.e. the number of element of $A(\mathbf{t})$) we feed to the regression also effects the result.

So we come up with an adaptive scheme: feed $A(\mathbf{t}')$ to the regression model and calculate $s(\mathbf{t'})$, if $\max(s(\mathbf{t}'))>tol$, reduce the size of $\mathbf{t}'$ by $\tau$ and repeat until $\max(s(\mathbf{t}'))<tol$ is satisfied. The result $\mathbf{t}'$ is the region that we believe $A(t)$ decays with time exponentially. 

% The \nocite command causes all entries in a bibliography to be printed out
% whether or not they are actually referenced in the text. This is appropriate
% for the sample file to show the different styles of references, but authors
% most likely will not want to use it.
%\nocite{*}

\selectlanguage{french}
\bibliography{MCL}% Produces the bibliography via BibTeX.

\end{document}
%
% ****** End of file apssamp.tex ******
